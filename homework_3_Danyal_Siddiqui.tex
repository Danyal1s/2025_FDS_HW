% Options for packages loaded elsewhere
\PassOptionsToPackage{unicode}{hyperref}
\PassOptionsToPackage{hyphens}{url}
%
\documentclass[
]{article}
\usepackage{amsmath,amssymb}
\usepackage{iftex}
\ifPDFTeX
  \usepackage[T1]{fontenc}
  \usepackage[utf8]{inputenc}
  \usepackage{textcomp} % provide euro and other symbols
\else % if luatex or xetex
  \usepackage{unicode-math} % this also loads fontspec
  \defaultfontfeatures{Scale=MatchLowercase}
  \defaultfontfeatures[\rmfamily]{Ligatures=TeX,Scale=1}
\fi
\usepackage{lmodern}
\ifPDFTeX\else
  % xetex/luatex font selection
\fi
% Use upquote if available, for straight quotes in verbatim environments
\IfFileExists{upquote.sty}{\usepackage{upquote}}{}
\IfFileExists{microtype.sty}{% use microtype if available
  \usepackage[]{microtype}
  \UseMicrotypeSet[protrusion]{basicmath} % disable protrusion for tt fonts
}{}
\makeatletter
\@ifundefined{KOMAClassName}{% if non-KOMA class
  \IfFileExists{parskip.sty}{%
    \usepackage{parskip}
  }{% else
    \setlength{\parindent}{0pt}
    \setlength{\parskip}{6pt plus 2pt minus 1pt}}
}{% if KOMA class
  \KOMAoptions{parskip=half}}
\makeatother
\usepackage{xcolor}
\usepackage[margin=1in]{geometry}
\usepackage{color}
\usepackage{fancyvrb}
\newcommand{\VerbBar}{|}
\newcommand{\VERB}{\Verb[commandchars=\\\{\}]}
\DefineVerbatimEnvironment{Highlighting}{Verbatim}{commandchars=\\\{\}}
% Add ',fontsize=\small' for more characters per line
\usepackage{framed}
\definecolor{shadecolor}{RGB}{248,248,248}
\newenvironment{Shaded}{\begin{snugshade}}{\end{snugshade}}
\newcommand{\AlertTok}[1]{\textcolor[rgb]{0.94,0.16,0.16}{#1}}
\newcommand{\AnnotationTok}[1]{\textcolor[rgb]{0.56,0.35,0.01}{\textbf{\textit{#1}}}}
\newcommand{\AttributeTok}[1]{\textcolor[rgb]{0.13,0.29,0.53}{#1}}
\newcommand{\BaseNTok}[1]{\textcolor[rgb]{0.00,0.00,0.81}{#1}}
\newcommand{\BuiltInTok}[1]{#1}
\newcommand{\CharTok}[1]{\textcolor[rgb]{0.31,0.60,0.02}{#1}}
\newcommand{\CommentTok}[1]{\textcolor[rgb]{0.56,0.35,0.01}{\textit{#1}}}
\newcommand{\CommentVarTok}[1]{\textcolor[rgb]{0.56,0.35,0.01}{\textbf{\textit{#1}}}}
\newcommand{\ConstantTok}[1]{\textcolor[rgb]{0.56,0.35,0.01}{#1}}
\newcommand{\ControlFlowTok}[1]{\textcolor[rgb]{0.13,0.29,0.53}{\textbf{#1}}}
\newcommand{\DataTypeTok}[1]{\textcolor[rgb]{0.13,0.29,0.53}{#1}}
\newcommand{\DecValTok}[1]{\textcolor[rgb]{0.00,0.00,0.81}{#1}}
\newcommand{\DocumentationTok}[1]{\textcolor[rgb]{0.56,0.35,0.01}{\textbf{\textit{#1}}}}
\newcommand{\ErrorTok}[1]{\textcolor[rgb]{0.64,0.00,0.00}{\textbf{#1}}}
\newcommand{\ExtensionTok}[1]{#1}
\newcommand{\FloatTok}[1]{\textcolor[rgb]{0.00,0.00,0.81}{#1}}
\newcommand{\FunctionTok}[1]{\textcolor[rgb]{0.13,0.29,0.53}{\textbf{#1}}}
\newcommand{\ImportTok}[1]{#1}
\newcommand{\InformationTok}[1]{\textcolor[rgb]{0.56,0.35,0.01}{\textbf{\textit{#1}}}}
\newcommand{\KeywordTok}[1]{\textcolor[rgb]{0.13,0.29,0.53}{\textbf{#1}}}
\newcommand{\NormalTok}[1]{#1}
\newcommand{\OperatorTok}[1]{\textcolor[rgb]{0.81,0.36,0.00}{\textbf{#1}}}
\newcommand{\OtherTok}[1]{\textcolor[rgb]{0.56,0.35,0.01}{#1}}
\newcommand{\PreprocessorTok}[1]{\textcolor[rgb]{0.56,0.35,0.01}{\textit{#1}}}
\newcommand{\RegionMarkerTok}[1]{#1}
\newcommand{\SpecialCharTok}[1]{\textcolor[rgb]{0.81,0.36,0.00}{\textbf{#1}}}
\newcommand{\SpecialStringTok}[1]{\textcolor[rgb]{0.31,0.60,0.02}{#1}}
\newcommand{\StringTok}[1]{\textcolor[rgb]{0.31,0.60,0.02}{#1}}
\newcommand{\VariableTok}[1]{\textcolor[rgb]{0.00,0.00,0.00}{#1}}
\newcommand{\VerbatimStringTok}[1]{\textcolor[rgb]{0.31,0.60,0.02}{#1}}
\newcommand{\WarningTok}[1]{\textcolor[rgb]{0.56,0.35,0.01}{\textbf{\textit{#1}}}}
\usepackage{graphicx}
\makeatletter
\def\maxwidth{\ifdim\Gin@nat@width>\linewidth\linewidth\else\Gin@nat@width\fi}
\def\maxheight{\ifdim\Gin@nat@height>\textheight\textheight\else\Gin@nat@height\fi}
\makeatother
% Scale images if necessary, so that they will not overflow the page
% margins by default, and it is still possible to overwrite the defaults
% using explicit options in \includegraphics[width, height, ...]{}
\setkeys{Gin}{width=\maxwidth,height=\maxheight,keepaspectratio}
% Set default figure placement to htbp
\makeatletter
\def\fps@figure{htbp}
\makeatother
\setlength{\emergencystretch}{3em} % prevent overfull lines
\providecommand{\tightlist}{%
  \setlength{\itemsep}{0pt}\setlength{\parskip}{0pt}}
\setcounter{secnumdepth}{-\maxdimen} % remove section numbering
\ifLuaTeX
  \usepackage{selnolig}  % disable illegal ligatures
\fi
\usepackage{bookmark}
\IfFileExists{xurl.sty}{\usepackage{xurl}}{} % add URL line breaks if available
\urlstyle{same}
\hypersetup{
  pdftitle={Homework 3},
  pdfauthor={GSND 5345Q, Fundamentals of Data Science},
  hidelinks,
  pdfcreator={LaTeX via pandoc}}

\title{Homework 3}
\author{GSND 5345Q, Fundamentals of Data Science}
\date{Due Monday, February 10th, 2025}

\begin{document}
\maketitle

\subsection*{GitHub (50 points)}

In this section, you will reinforce the concepts and skills covered in
the introductory lecture on Git and GitHub. You will answer a set of
conceptual questions to demonstrate your understanding of Git's core
functionality and complete hands-on tasks to practice using Git commands
and interacting with GitHub.

\begin{enumerate}

  \item What is the purpose of Git, and how does it differ from GitHub?

  \item Answer: Git is an open-source software and version control system that allows for tracking of file versions, which is useful for documenting changes to code. GitHub is a platform for hosting and sharing Git code with other users online via repositories.

  \item Explain the difference between a commit and a push in Git.
  \item What does the command git clone do, and how is it different from git pull?
  \item Initialize a new Git repository in a local folder and create a file named README.md. Add some text to it, commit the changes, and push it to a new GitHub repository. Submit the link to your repository.
        \item Fork the \href{https://github.com/wevanjohnson/my.package}{https://github.com/wevanjohnson/my.package} directory and clone it to your local machine. Then add your name as an author in the DESCRIPTION file local repository and add a multiplication function to the R package (R folder). Then push the changes to your GitHub fork, and send me a pull request with your changes.  
        \item Clone the \href{https://github.com/wevanjohnson/2025\_Spring\_FDS}{https://github.com/wevanjohnson/2025\_Spring\_FDS} repository on your computer. Find something that could be improved (typo? explain somthing better), add files/changes to it, and upload it to GitHub. Send another well-annotated pull request to Dr. Johnson. 
    \end{enumerate}

\subsection*{R Basics (50 points)}

These exercises will give you some introductory experience with the R
programming basics. Please complete the following:

1. What is the sum of the first 100 positive integers? The formula for
the sum of integers \(1\) through \(n\) is \(n(n+1)/2\). Define
\(n=100\) and then use R to compute the sum of \(1\) through \(100\)
using the formula. What is the sum?

2. Now use the same formula to compute the sum of the integers from 1
through 1,000.

3. Look at the result of typing the following code into R:

\begin{Shaded}
\begin{Highlighting}[]
\NormalTok{n }\OtherTok{\textless{}{-}} \DecValTok{1000}
\NormalTok{x }\OtherTok{\textless{}{-}} \FunctionTok{seq}\NormalTok{(}\DecValTok{1}\NormalTok{, n)}
\FunctionTok{sum}\NormalTok{(x)}
\end{Highlighting}
\end{Shaded}

Based on the result, what do you think the functions \texttt{seq} and
\texttt{sum} do? You can use \texttt{help}.

\begin{enumerate}
\def\labelenumi{\alph{enumi}.}
\tightlist
\item
  \texttt{sum} creates a list of numbers and \texttt{seq} adds them up.
\item
  \texttt{seq} creates a list of numbers and \texttt{sum} adds them up.
\item
  \texttt{seq} creates a random list and \texttt{sum} computes the sum
  of 1 through 1,000.
\item
  \texttt{sum} always returns the same number.
\end{enumerate}

4. In math and programming, we say that we evaluate a function when we
replace the argument with a given number. So if we type
\texttt{sqrt(4)}, we evaluate the \texttt{sqrt} function. In R, you can
evaluate a function inside another function. The evaluations happen from
the inside out. Use one line of code to compute the log, in base 10, of
the square root of 100.

5. Which of the following will always return the numeric value stored in
\texttt{x}? You can try out examples and use the help system if you
want.

\begin{enumerate}
\def\labelenumi{\alph{enumi}.}
\tightlist
\item
  \texttt{log(10\^{}x)}
\item
  \texttt{log10(x\^{}10)}
\item
  \texttt{log(exp(x))}
\item
  \texttt{exp(log(x,\ base\ =\ 2))}
\end{enumerate}

6. Load the US murders dataset.

\begin{Shaded}
\begin{Highlighting}[]
\FunctionTok{library}\NormalTok{(dslabs)}
\FunctionTok{data}\NormalTok{(murders)}
\end{Highlighting}
\end{Shaded}

Use the function \texttt{str} to examine the structure of the
\texttt{murders} object. Which of the following best describes the
variables represented in this data frame?

\begin{enumerate}
\def\labelenumi{\alph{enumi}.}
\tightlist
\item
  The 51 states.
\item
  The murder rates for all 50 states and DC.
\item
  The state name, the abbreviation of the state name, the state's
  region, and the state's population and total number of murders for
  2010.
\item
  \texttt{str} shows no relevant information.
\end{enumerate}

7. What are the column names used by the data frame for these five
variables?

8. Use the accessor \texttt{\$} to extract the state abbreviations and
assign them to the object \texttt{a}. What is the class of this object?

9. Now use the square brackets to extract the state abbreviations and
assign them to the object \texttt{b}. Use the \texttt{identical}
function to determine if \texttt{a} and \texttt{b} are the same.

10. We saw that the \texttt{region} column stores a factor. You can
corroborate this by typing:

\begin{Shaded}
\begin{Highlighting}[]
\FunctionTok{class}\NormalTok{(murders}\SpecialCharTok{$}\NormalTok{region)}
\end{Highlighting}
\end{Shaded}

With one line of code, use the function \texttt{levels} and
\texttt{length} to determine the number of regions defined by this
dataset.

11. The function \texttt{table} takes a vector and returns the frequency
of each element. You can quickly see how many states are in each region
by applying this function. Use this function in one line of code to
create a table of states per region.

12. Use the function \texttt{c} to create a vector with the average high
temperatures in January for Beijing, Lagos, Paris, Rio de Janeiro, San
Juan, and Toronto, which are 35, 88, 42, 84, 81, and 30 degrees
Fahrenheit. Call the object \texttt{temp}.

13. Now create a vector with the city names and call the object
\texttt{city}.

14. Use the \texttt{names} function and the objects defined in the
previous exercises to associate the temperature data with its
corresponding city.

15. Use the \texttt{{[}} and \texttt{:} operators to access the
temperature of the first three cities on the list.

16. Use the \texttt{{[}} operator to access the temperature of Paris and
San Juan.

17. Use the \texttt{:} operator to create a sequence of numbers
\(12,13,14,\dots,73\).

18. Create a vector containing all the positive odd numbers smaller than
100.

19. Create a vector of numbers that starts at 6, does not pass 55, and
adds numbers in increments of 4/7: 6, 6 + 4/7, 6 + 8/7, and so on. How
many numbers does the list have? Hint: use \texttt{seq} and
\texttt{length}.

20. What is the class of the following object
\texttt{a\ \textless{}-\ seq(1,\ 10,\ 0.5)}?

These exercises will give you some introductory experience with
programming basics. Please complete the following:

21. What will this conditional expression return?

\begin{Shaded}
\begin{Highlighting}[]
\NormalTok{x }\OtherTok{\textless{}{-}} \FunctionTok{c}\NormalTok{(}\DecValTok{1}\NormalTok{,}\DecValTok{2}\NormalTok{,}\SpecialCharTok{{-}}\DecValTok{3}\NormalTok{,}\DecValTok{4}\NormalTok{)}

\ControlFlowTok{if}\NormalTok{(}\FunctionTok{all}\NormalTok{(x}\SpecialCharTok{\textgreater{}}\DecValTok{0}\NormalTok{))\{}
  \FunctionTok{print}\NormalTok{(}\StringTok{"All Postives"}\NormalTok{)}
\NormalTok{\} }\ControlFlowTok{else}\NormalTok{\{}
  \FunctionTok{print}\NormalTok{(}\StringTok{"Not all positives"}\NormalTok{)}
\NormalTok{\}}
\end{Highlighting}
\end{Shaded}

22. Which of the following expressions is always \texttt{FALSE} when at
least one entry of a logical vector \texttt{x} is TRUE?

\begin{enumerate}
\def\labelenumi{\alph{enumi}.}
\tightlist
\item
  \texttt{all(x)}
\item
  \texttt{any(x)}
\item
  \texttt{any(!x)}
\item
  \texttt{all(!x)}
\end{enumerate}

23. Create a function \texttt{sum\_n} that for any given value, say
\(n\), computes the sum of the integers from 1 to n (inclusive). Use the
function to determine the sum of integers from 1 to 5,000.

24. After running the code below, what is the value of \texttt{x}?

\begin{Shaded}
\begin{Highlighting}[]
\NormalTok{x }\OtherTok{\textless{}{-}} \DecValTok{3}
\NormalTok{my\_func }\OtherTok{\textless{}{-}} \ControlFlowTok{function}\NormalTok{(y)\{}
\NormalTok{  x }\OtherTok{\textless{}{-}} \DecValTok{5}
\NormalTok{  y}\SpecialCharTok{+}\DecValTok{5}
\NormalTok{\}}
\end{Highlighting}
\end{Shaded}

25. Write a function \texttt{compute\_s\_n} that for any given \(n\)
computes the sum \(S_n = 1^2 + 2^2 + 3^2 + \dots n^2\). Report the value
of the sum when \(n=10\).

\end{document}
